\documentclass[a4paper,12pt]{article}
%%%%%%%%%%%%%%%%%%%%%%%%%%%%%%%%%%%%%%%%%%%%%%%%%%%%%%%%%%%%%%%%%%%%%%%%%%%%%%%%%%%%%%%%%%%%%%%%%%%%%%%%%%%%%%%%%%%%%%%%%%%%%%%%%%%%%%%%%%%%%%%%%%%%%%%%%%%%%%%%%%%%%%%%%%%%%%%%%%%%%%%%%%%%%%%%%%%%%%%%%%%%%%%%%%%%%%%%%%%%%%%%%%%%%%%%%%%%%%%%%%%%%%%%%%%%
\usepackage{eurosym}
\usepackage{vmargin}
\usepackage{amsmath}
\usepackage{graphics}
\usepackage{epsfig}
\usepackage{subfigure}
\usepackage{framed}
\usepackage{enumerate}
\usepackage{fancyhdr}

\setcounter{MaxMatrixCols}{10}
%TCIDATA{OutputFilter=LATEX.DLL}
%TCIDATA{Version=5.00.0.2570}
%TCIDATA{<META NAME="SaveForMode"CONTENT="1">}
%TCIDATA{LastRevised=Wednesday, February 23, 201113:24:34}
%TCIDATA{<META NAME="GraphicsSave" CONTENT="32">}
%TCIDATA{Language=American English}

\pagestyle{fancy}
\setmarginsrb{20mm}{0mm}{20mm}{25mm}{12mm}{11mm}{0mm}{11mm}
\lhead{MS4222} \rhead{Kevin O'Brien} \chead{Week 5 Revision Class} %\input{tcilatex}
%%%%%%%%%%%%%%%%%%%%%%%%%%%%%%%%%%%%%%%%%%%%%
\begin{document}
\begin{enumerate}

\item Compute the range of the following Data Set.

\[\{65,73,89,56,73,52,47\}\]

\noindent \textbf{Solution:}\\
The range of $\{65,73,89,56,73,52,47\}$ is $ 89-47 = 42$.

\item 
Suppose we have a data set $X$ comprised of the following values. 
		\[X = \{4,6,12,8,15,19,11, 13\}\]
		Calculate the arithmetic mean.\\
\noindent \textbf{Solution:}\\
		\begin{itemize}
		\item To compute $\bar{x}$, the \textbf{arithmetic mean} of X, we add up all the elements of $X$ and divide by the number of elements in $X$ (i.e. the sample size $n$).
		
		\[ \bar{x} = \frac{4+6+12+8+15+19+11+13}{8} =\frac{88}{8} = 11\] 
		\item We would verbalize $\bar{x}$ as ``x-bar".
	\end{itemize}

\item 
	
	Suppose we roll a die 8 times and get the following scores: $x = \{ 5, 2, 1, 6, 3, 5, 3, 1\}$ \\ 
	What is the sample mean of the scores $\bar{x}$?
	
	
\noindent \textbf{Solution:}\\	
	\[ \bar{x}  = {5 + 2 +  1 +  6 +  3 +  5 +  3 +  1 \over 8 } = {26 \over 8} =  3.25 \]

\item Compute the median of the following data sets.
	\begin{itemize}
		\item[(a)] 
		\[3, 13, 7, 5, 21, 23, 39, 23, 40, 23, 14, 12, 56, 23, 29\]
		\item[(g)] 
		\[10, 19, 10, 9, 3, 59, 22, 43, 18, 21, 21, 25, 5, 9\]
	\end{itemize}
\noindent \textbf{Solution:}\\
For the first exercise, the ordered data set is:
\[ \{3, 5, 7, 12, 13, 14, 21, 23, 23, 23, 23, 29, 39, 40, 56\}\]
There are 15 items in the data set, an odd sized data set. The median is the 8th item, which is 23.
	
For the second exercise, the ordered data set is:
\[ \{3, 5, 9, 9, 10, 10, 18, 19, 21, 21, 22, 25, 43, 59\}\]
There are 14 items in the data set, an even sized data set. The median is the average of the 7th and 8th items, i.e. 18 and 19, which is 18.5.


\item 
	The duration, in months, of the construction phase of a number of motorway projects
	were collected and tabulated as follows
	
	\[ \{49, 55, 43, 45, 41, 33, 42 \} \]
	\begin{enumerate}[(a)]
	\item Calculate the mean value of the durations.
	\item Calculate the variance of the data set.
	\item Calculate the standard deviation.
%	\item Calculate the coefficient of variation.
	\end{enumerate}
	
\noindent \textbf{Solution:}
\begin{itemize}
    \item Firstly, what is the sample size? Answer: n = 7.
    \[ \bar{x} = {49 + 55 + 43 + 45 + 41 +33 + 42 \over 7} \]

	
	
	
	
	
\item The sample mean $\bar{x}$ is therefore	
	\[\bar{x} = 44.\]

\item The sample variance is computed as follows:
	
	\[s^2  = \frac{(49-44)^2 + (55-44)^2 + (43-44)^2 + (45-44)^2 + (41-44)^2 + (33-44)^2 + (42-44)^2} { 7-1}\]
	\[
	\frac{25 + 121 + 1 + 1 + 9 + 121 + 4}{6}
	= \frac{282}{6} = 47\]
	\[s^2 = 47\]	
\item The sample standard deviation is the square root of the variance.	
	\[s = \sqrt{47}\]
	
%\[cv= {s \over \bar{x}} \times 100\%	= 15.58\% \]

\end{itemize}
	

	
\end{enumerate}

\end{document}
