\documentclass[a4paper,12pt]{article}
%%%%%%%%%%%%%%%%%%%%%%%%%%%%%%%%%%%%%%%%%%%%%%%%%%%%%%%%%%%%%%%%%%%%%%%%%%%%%%%%%%%%%%%%%%%%%%%%%%%%%%%%%%%%%%%%%%%%%%%%%%%%%%%%%%%%%%%%%%%%%%%%%%%%%%%%%%%%%%%%%%%%%%%%%%%%%%%%%%%%%%%%%%%%%%%%%%%%%%%%%%%%%%%%%%%%%%%%%%%%%%%%%%%%%%%%%%%%%%%%%%%%%%%%%%%%
\usepackage{eurosym}
\usepackage{vmargin}
\usepackage{amsmath}
\usepackage{graphics}
\usepackage{epsfig}
\usepackage{subfigure}
\usepackage{enumerate}
\usepackage{fancyhdr}

\setcounter{MaxMatrixCols}{10}
%TCIDATA{OutputFilter=LATEX.DLL}
%TCIDATA{Version=5.00.0.2570}
%TCIDATA{<META NAME="SaveForMode"CONTENT="1">}
%TCIDATA{LastRevised=Wednesday, February 23, 201113:24:34}
%TCIDATA{<META NAME="GraphicsSave" CONTENT="32">}
%TCIDATA{Language=American English}

\pagestyle{fancy}
\setmarginsrb{20mm}{0mm}{20mm}{25mm}{12mm}{11mm}{0mm}{11mm}
\lhead{MS4222} \rhead{Kevin O'Brien} \chead{Descriptive Statistics} %\input{tcilatex}

\begin{document}


\section*{Median}
	\begin{itemize}
		\item As well as the mean, another commonly used measure of centrality is the median.
		
		\item The median is the value halfway through the ordered data set, below and above which there lies an equal number of data values.
		\begin{itemize}
		\item[$\ast$] For an odd sized data set, the median is the middle element of the \textbf{ordered} data set.
		\item[$\ast$] For an even sized data set, the median is the average of the middle pair of elements of an \textbf{ordered} data set.
		\end{itemize}
		\item It is generally a good descriptive measure of the location which works well for \textbf{\emph{skewed data}}, or data with \textbf{\emph{outliers}}.
		
		\item For later, the median is the 0.5 quantile, and the second quartile $Q_2$.
	\end{itemize}

\subsection*{Computing the Median}

			
			
			With an odd number of data values, for example nine, we have: 
			\begin{itemize}
				\item Data  $\{96, 48, 27, 72, 39, 70, 7, 68, 99 \}$
				\item Ordered Data  $\{7, 27, 39, 48, 68, 70, 72, 96, 99\}$
				\item The Median is 68, leaving four values below and four values above 
			\end{itemize}
			\bigskip
			With an even number of data values, for example 8, we have: 
			\begin{itemize}
				\item Data  $\{96, 48 ,27 ,72, 39, 70, 7, 68  \}$
				\item Ordered Data  $\{7, 27, 39, 48, 68, 70, 72, 96\}$
				\item The median is halfway between the two 'middle' data points - in this case halfway between 48 and 68. Therefore the median is 58 
			\end{itemize}

	\end{document}
