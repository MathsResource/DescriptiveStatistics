
\documentclass[]{article}

\voffset=-1.5cm
\oddsidemargin=0.0cm
\textwidth = 480pt

\usepackage{framed}
\usepackage{subfiles}
\usepackage{graphics}
\usepackage{newlfont}
\usepackage{eurosym}
\usepackage{amsmath,amsthm,amsfonts}
\usepackage{amsmath}
\usepackage{enumerate}
\usepackage{color}
\usepackage{multicol}
\usepackage{amssymb}
\usepackage{multicol}
\usepackage[dvipsnames]{xcolor}
\usepackage{graphicx}
\begin{document}

\section*{MS4222 Week 3 Question 10} 

Two dice are rolled, find the probability that the sum is 


(a) equal to 1 

(b) equal to 4 

(c) less than 13



\noindent \textbf{Solution to Question 10}

\noindent The sample space S of two dice is shown below. 

\begin{framed}
\begin{verbatim}
  S = { (1,1),(1,2),(1,3),(1,4),(1,5),(1,6) 
        (2,1),(2,2),(2,3),(2,4),(2,5),(2,6) 
        (3,1),(3,2),(3,3),(3,4),(3,5),(3,6) 
        (4,1),(4,2),(4,3),(4,4),(4,5),(4,6) 
        (5,1),(5,2),(5,3),(5,4),(5,5),(5,6) 
        (6,1),(6,2),(6,3),(6,4),(6,5),(6,6) } 
\end{verbatim}
Important: Each sample point is equally probable. If this was not the case, then the following approach is invalid.
\end{framed}

\noindent \textbf{Notation}
\begin{itemize}
    \item $n(E)$ Number of sample points in event $E$
    \item $n(S)$ Number of sample points in samples space $S$
    \item $P(E)$ Probability of event $E$
\end{itemize}


\begin{enumerate}[(a)]
\item Let E be the event ``sum equal to 1". There are no outcomes which correspond to a sum equal to 1, hence 

\[P(E) = n(E) / n(S) = 0 / 36 = 0 \]

\item Three possible outcomes give a sum equal to 4: here $E = \{(1,3),(2,2),(3,1)\}$, hence. 

\[P(E) = n(E) / n(S) = 3 / 36 = 1 / 12 \]

\item All possible outcomes, $E = S$, give a sum less than 13, hence. 

\[P(E) = n(E) / n(S) = 36 / 36 = 1 \]
\end{enumerate}
%==============================================================================%
\end{document}
