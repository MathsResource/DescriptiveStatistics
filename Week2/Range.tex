\documentclass[a4paper,12pt]{article}
%%%%%%%%%%%%%%%%%%%%%%%%%%%%%%%%%%%%%%%%%%%%%%%%%%%%%%%%%%%%%%%%%%%%%%%%%%%%%%%%%%%%%%%%%%%%%%%%%%%%%%%%%%%%%%%%%%%%%%%%%%%%%%%%%%%%%%%%%%%%%%%%%%%%%%%%%%%%%%%%%%%%%%%%%%%%%%%%%%%%%%%%%%%%%%%%%%%%%%%%%%%%%%%%%%%%%%%%%%%%%%%%%%%%%%%%%%%%%%%%%%%%%%%%%%%%
\usepackage{eurosym}
\usepackage{vmargin}
\usepackage{amsmath}
\usepackage{graphics}
\usepackage{epsfig}
\usepackage{subfigure}
\usepackage{framed}
\usepackage{enumerate}
\usepackage{fancyhdr}

\setcounter{MaxMatrixCols}{10}
%TCIDATA{OutputFilter=LATEX.DLL}
%TCIDATA{Version=5.00.0.2570}
%TCIDATA{<META NAME="SaveForMode"CONTENT="1">}
%TCIDATA{LastRevised=Wednesday, February 23, 201113:24:34}
%TCIDATA{<META NAME="GraphicsSave" CONTENT="32">}
%TCIDATA{Language=American English}

\pagestyle{fancy}
\setmarginsrb{20mm}{0mm}{20mm}{25mm}{12mm}{11mm}{0mm}{11mm}
\lhead{MS4222} \rhead{Kevin O'Brien} \chead{Descriptive Statistics} %\input{tcilatex}

\begin{document}
	\section*{Range}


\noindent The simplest measure of dispersion is the range.
The range is simply the difference of the lowest and highest values. The range is
another easy-to-understand measure, but it will clearly be very affected by a few
extreme values.	
	\begin{itemize}
		\item The range of a sample (or a data set) is a measure of the spread or the dispersion of the observations. \item It is the difference between the largest and the smallest observed value of some quantitative characteristic and is very easy to calculate.
		\[ \operatorname{Range} = \operatorname{Max} -\operatorname{Min} \]
		
		
		\item A great deal of information is ignored when computing the range since only the largest and the smallest data values are considered; the remaining data are ignored.
		
		\item The range value of a data set is greatly influenced by the presence of just one unusually large or small value in the sample (outlier).
	\end{itemize}
	
	\noindent \textbf{Example}
	
	
	The range of $\{65,73,89,56,73,52,47\}$ is $ 89-47 = 42$.
	
	% If the highest score in a 1st year statistics exam was 98 and the lowest 48, then the range would be 98-48 = 50.
	

\end{document}
