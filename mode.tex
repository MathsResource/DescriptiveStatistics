The \textbf{mode} is the most frequently occurring value in the set of scores. 
To determine the mode, you might again order the scores as shown above, and then count each one. 
The most frequently occurring value is the mode. In our example, the value 15 occurs three times 
and is the model. In some distributions there is more than one modal value. For instance, in a 
bimodal distribution there are two values that occur most frequently.


3.  The mode of a set of data (from a sample or a population) is defined as the number which occurs  with greatest frequency

Example  
Given the data that follows
32  42  46 46  54

The mode is 46 as this occurs more often than any other data value.
Situations can arise for which the greatest frequency occurs at two or more different values.
If the data have two modes, the data is said to be bimodal.  If the data have more than two modes, the data are said to be multimodal.
