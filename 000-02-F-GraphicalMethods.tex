
\documentclass[]{report}

\voffset=-1.5cm
\oddsidemargin=0.0cm
\textwidth = 480pt

\usepackage{framed}
\usepackage{subfiles}
\usepackage{graphics}
\usepackage{newlfont}
\usepackage{eurosym}
\usepackage{amsmath,amsthm,amsfonts}
\usepackage{amsmath}
\usepackage{color}
\usepackage{amssymb}
\usepackage{multicol}
\usepackage[dvipsnames]{xcolor}
\usepackage{graphicx}
\begin{document}

\chapter{Graphical Methods}

{
\subsection{Graphical Procedures for Statistics}
\begin{itemize}
\item Bar-plots
\item Histograms
\item Boxplots

\end{itemize}
}
{
\textbf{Today's Class}

\begin{itemize}
\item More on Graphical methods
\begin{itemize}
\item Bar charts
\item Box-and-whisker plots
\end{itemize}
\item Discrete probability distributions
\begin{itemize}
\item Binomial Experiments
\item The Binomial Probability distribution
\end{itemize}
\end{itemize}
}

\newpage
\subsection{Bar plots}
\large
\begin{itemize} \item A bar plot displays the frequency (or relative frequency) for all observations of a discrete random variable. \item A bar plot is much like a histogram, in that the heights of columns represent the frequency (or relative frequency) of each outcome.
\item Each outcome of a random experiment corresponds to one and only one column of the bar plot.
\item A bar plot differs from a histogram in that the columns are distinct and separated from each other by a small distance.
\end{itemize}

\subsection{Bar plots}
\large
Suppose we roll a die 300 times, and obtain the following results

\begin{center}
\begin{tabular}{|c|c|c|c|c|c|c|}
\hline

Outcome & 1 & 2 & 3 & 4 & 5 & 6 \\
Frequency & 59 &41 &39 &52 &57 &52  \\
Rel. Freq & 0.196 & 0.136 & 0.130 & 0.173 & 0.190 & 0.173\\
\hline
\end{tabular}
\end{center}


{
\subsection{Bar Plots}

\begin{center}
\includegraphics[scale=0.40]{images/3Bbarplot2}
\end{center}
}


%---------------------------------------------------------------------------%


%------------------------------------------------------------------%
{
\textbf{Bar plots}

\begin{itemize} \item A bar plot displays the frequency (or relative frequency) for all observations of a discrete random variable. \item A bar plot is much like a histogram, in that the heights of columns represent the frequency (or relative frequency) of each outcome.
\item Each outcome of a random experiment corresponds to one and only one column of the bar plot.
\item A bar plot differs from a histogram in that the columns are distinct and separated from each other by a small distance.
\end{itemize}
}
%------------------------------------------------------------------%
{
\textbf{Bar plots}

Suppose we roll a die 300 times, and obtain the following results

\begin{center}
\begin{tabular}{|c|c|c|c|c|c|c|}
\hline

Outcome & 1 & 2 & 3 & 4 & 5 & 6 \\
Frequency & 59 &41 &39 &52 &57 &52  \\
Rel. Freq & 0.196 & 0.136 & 0.130 & 0.173 & 0.190 & 0.173\\
\hline
\end{tabular}
\end{center}

On the next slide is the bar plot of the relative frequencies of the outcomes of die throw experiment.
Included on the bar plot is the theoretical probability of each outcome. As each outcome is equally probable, this is just a straight line.\\ \bigskip
Minor deviations from the theoretical probability can often be assumed to be as a result of random error. In the case of large deviations, there may be a flawed assumption about the theoretical probabilities.
}

%--------------------------------------------------------%

{
\textbf{Relative Frequency Bar Plots}

\begin{center}
\includegraphics[scale=0.40]{images/3Bbarplot}
\end{center}
}

%--------------------------------------------------------%

{
\textbf{Relative Frequency Bar Plots}

\begin{itemize}
\item Just as bar plots can be used to graphically depict observed relative frequencies, they can be used to
depict the theoretical probabilities of each outcome.
\item We will be using bar plots to visualize the theoretical probabilities of outcomes of discrete random variables.
\item For this module, bar plots are assumed to be used for this purpose, unless it is clearly expressed otherwise.
\item On the next slide is a bar plot of the probabilities of each outcome of a dice throw.
\end{itemize}

}


%--------------------------------------------------------%

{
\textbf{Bar Plots}
\begin{center}
\includegraphics[scale=0.40]{images/3Bbarplot3}
\end{center}
}

{
\textbf{Bar Plots}

\begin{itemize}
\item Bar plots are useful in that they visualize `events'.
\item Consider the event where either a `4' or a `5' is thrown.
\item The relevant columns for this event are shaded (next slide).
\item We will be using bar plots for depicting specific events in upcoming material
\end{itemize}

}


%---------------------------------------------------------------------------%
%------------------------------------------------------------------%
{
\textbf{Bar plots}

\begin{itemize} \item  A bar plot displays the frequency (or relative frequency) for all observations of a discrete random variable. \item  A bar plot is much like a histogram, in that the heights of columns represent the frequency (or relative frequency) of each outcome.
\item  Each outcome of a random experiment corresponds to one and only one column of the bar plot.
\item  A bar plot differs from a histogram in that the columns are distinct and separated from each other by a small distance.
\end{itemize}
}
%------------------------------------------------------------------%
{
\textbf{Bar plots}

Suppose we roll a die 300 times, and obtain the following results

\begin{center}
\begin{tabular}{|c|c|c|c|c|c|c|}
\hline

Outcome & 1 & 2 & 3 & 4 & 5 & 6 \\
Frequency & 59 &41 &39 &52 &57 &52  \\
Rel. Freq & 0.196 & 0.136 & 0.130 & 0.173 & 0.190 & 0.173\\
\hline
\end{tabular}
\end{center}

On the next slide is the bar plot of the relative frequencies of the outcomes of die throw experiment.
Included on the bar plot is the theoretical probability of each outcome. As each outcome is equally probable, this is just a straight line.\\ \bigskip
Minor deviations from the theoretical probability can often be assumed to be as a result of random error. In the case of large deviations, there may be a flawed assumption about the theoretical probabilities.
}

%--------------------------------------------------------%

{
\textbf{Relative Frequency Bar Plots}

\begin{center}
\includegraphics[scale=0.40]{images/3Bbarplot}
\end{center}
\begin{itemize}
\item  Just as bar plots can be used to graphically depict observed relative frequencies, they can be used to
depict the theoretical probabilities of each outcome.
\item  We will be using bar plots to visualize the theoretical probabilities of outcomes of discrete random variables.
\item  For this module, bar plots are assumed to be used for this purpose, unless it is clearly expressed otherwise.
\item  On the next slide is a bar plot of the probabilities of each outcome of a dice throw.
\end{itemize}


\textbf{Bar Plots}
\begin{center}
\includegraphics[scale=0.40]{images/3Bbarplot3}
\end{center}
}

{
\textbf{Bar Plots}

\begin{itemize}
\item  Bar plots are useful in that they visualize `events'.
\item  Consider the event where either a `4' or a `5' is thrown.
\item  The relevant columns for this event are shaded (next slide).
\item  We will be using bar plots for depicting specific events in upcoming material
\end{itemize}

\newpage
\section{Bar Plots}

\begin{center}
\includegraphics[scale=0.40]{images/3Bbarplot2}
\end{center}

\textbf{Boxplots}
\begin{itemize}
\item  The second graphical method we will be looking at today is the `box-and-whisker' plot (commonly just referred to as `boxplots')
\item  The boxplots is a useful tool for assessing the distribution of a dataset, by means of a visual summary.
\item  Recall the data set of the exam scores of 100 students from yesterday's class (see next slide).
\item  The quartiles of the data set were $Q_1 = 42.5$, $Q_2 = 54.5$ (with $Q_2$ being the median), and $Q_3 =  65.5$ respectively.
\item  The interquartile range is $Q_3 - Q_1 = 23$
\item  The boxplot of the distribution is featured on the next slide.
\end{itemize}
}
%---------------------------------------------------------------------------%
{
\begin{table}[ht]
\caption{Exam results of 100 students} % title of Table
\centering % used for centering table
\begin{tabular}{|c ccc ccc ccc|} % centered columns (4 columns)\hline
\hline

13&21&22&23&24&25&26&28&29&30\\31&32&33&34&35& 36&36&36&37&38\\
39&41&41&41&42&43&44&44&44&45\\45&46&47&49&50& 51&51&52&53&53\\
53&53&53&54&54&54&54&54&54&54\\55&55&55&56&56& 56&57&57&58&59\\
62&63&63&63&63&64&64&64&64&64\\65&65&65&65&65& 66&66&66&67&69\\
71&71&72&72&73&74&75&76&76&76\\77&82&84&85&87& 88&91&91&92&99\\ \hline
\end{tabular}
\end{table}
}
%--------------------------------------------------------%

{
\textbf{Boxplots}

\begin{center}
\includegraphics[scale=0.40]{images/3Bboxplot1}
\end{center}
}

%--------------------------------------------------------%
{
\textbf{Boxplots}
\begin{itemize}
\item  The boxplot is a visual summary containing important aspects of a distribution. \item  The main component of the plot , the `\textbf{\emph{box}}', stretches from the \textbf{\emph{lower hinge}}, defined as $Q_1$, to the \textbf{\emph{upperhinge}}, defined as $Q_3$ .
\item  The median is shown as a line across the box.
\item  Therefore the box contains the middle half of the scores in the distribution.
\item   1/4 of the distribution is between the median line and the upper hinge. Similary 1/4 of the distribution is between the median line and the lower hinge.
\end{itemize}

\begin{itemize}
\item  On either side of the box are the \textbf{\emph{whiskers}}.
\item  To find where to place the whiskers, we must first compute the location of the \textbf{\emph{fences}}, and determine whether or not there are any \textbf{\emph{outliers}} present.
\item  Firstly, we must compute the location of the \textbf{\emph{lower fence}}.
\[ \mbox{ Lower Fence}  = Q_1 - 1.5 \times IQR \]
\item  For our example, the lower fence is
\[ \mbox{ Lower Fence}  = 42.5 - 1.5 \times 23  = 42.5 - 34.5 = 8 \]

\end{itemize}

\begin{itemize}
\item  The lower fence is used to determine whether there are any outliers in the lower half of the data set.
\item  If there is any observed value less than the lower fence, it is considered an outlier.
\item  The first whisker is drawn at the location of the lowest value that is not considered an outlier.
\item  If no values are considered outliers, then the whisker is drawn at the location of the smallest value of the dataset.
\item  For our dataset, the lowest value is 13, which is not less than the lower fence.
\item  Therefore we draw the first whisker , a vertical line, at this location.
\item  A horizontal line is drawn connecting the location of this whisker to $Q_1$.
\end{itemize}
}
{
\noindent \textbf{Outliers}
\begin{itemize}
\item  Any value considered to be an outlier should be indicated with an asterisk or a small circle.
\item  We will see an example of a boxplot with outliers in due course.
\end{itemize}


}
%---------------------------------------------------------------------------%
{
\textbf{Boxplots}
\begin{itemize}
\item  Now we must compute the location of the \textbf{\emph{upper fence}}.
\[ \mbox{ Upper Fence}  = Q_3 + 1.5 \times IQR \]
\item  For our example, the upper fence is
\[ \mbox{ Upper Fence}  = 65.5  + 1.5 \times 23  = 65.5 + 34.5 = 100 \]

\end{itemize}

\noindent \textbf{Upper Fence}
\begin{itemize}
\item  The upper fence is used to determine whether there are any outliers in the upper half of the data set.
\item  If there is any observed value greater than the upper fence, it is considered an outlier.
\item  The second whisker is drawn at the location of the highest value that is not considered an outlier.
\item  If no values are considered outliers, then the whisker is drawn at the location of the highest value of the dataset.
\item  For our dataset, the highest value is 99, which is less than the upper fence.
\item  Therefore we draw the second whisker , a vertical line, at this location.
\item  A horizontal line is drawn connecting the location of this whisker to $Q_3$.
\end{itemize}
}
%---------------------------------------------------------------------------%
{
\textbf{Boxplots}
\begin{itemize}
\item  Remark: If you do not get a sensible value for either the upper or lower fence, you can replace it with the nearest sensible value
\item  For example, suppose we got a negative lower fence value. It does not make sense to get a negative score in an exam.
\item  In this case, we could replace the value with a value of $0$.
\item  similarly for the upper fence: any fence value greater than 100 should be replaced with the value of 100.
\end{itemize}
}
%---------------------------------------------------------------------------%
{
\textbf{Using Boxplots to Compare Distributions}
\begin{itemize}
\item  Boxplots are very useful in comparing the distributions of two or more data sets.
\item  Recall the experiment of 60 students, each throwing a die 100 times.
\item  Suppose they perform this experiment twice, firstly with a fair die, and then with a crooked die.
\item  (The probability of the outcomes from the crooked die are as per yesterday's class).
\item  Boxplots can use used to compare the distribution of the outcomes of both experiments.
\end{itemize}


\begin{center}
\includegraphics[scale=0.40]{images/3Bboxplot1}
\end{center}
}

%--------------------------------------------------------------------------------------%
\newpage
%=====================================================%

{

\textbf{Graphical Procedures for Statistics}
\begin{itemize}
\item Bar-plots
\item Histograms
\item Boxplots

\end{itemize}
}
%----------------------------------------------------------------%
{
\textbf{Histograms}
\noindent \textbf{Thought Experiment}
\begin{itemize}
\item Consider an experiment in which each student in a class of 60 rolls a die 100 times.
\item Each score is recorded, and a total score is calculated.
\item As the expected value of rolled die is 3.5, the expected total is 350 for each student.
\item At the end of the experiment the students reported their totals.
\item The totals were put into ascending order, and tabulated as follows (next slide).
\end{itemize}

}

%--------------------------------------------------------%
{
\textbf{Outcomes of Die-Throw Experiment}
\small
\begin{center}
\begin{tabular}{|c c c c c c c c c c|}
\hline
% after \\: \hline or \cline{col1-col2} \cline{col3-col4} ...
307 & 321 & 324 & 328 & 329 & 330 & 334 & 335 & 336 &337 \\
337 & 337 & 338 & 339 & 339 & 342 & 343 & 343 & 344 &344 \\
346 & 346 & 347 & 348 & 348 & 348 & 350 & 351 & 352 &352 \\
353 & 353 & 353 & 354 & 354 & 356 & 356 & 357 & 357 &358 \\
358 & 360 & 360 & 361 & 362 & 363 & 365 & 365 & 369 &369 \\
370 & 370 & 374 & 378 & 381 & 384 & 385 & 386 & 392 &398 \\
\hline
\end{tabular}
\end{center}
\normalsize
\noindent {As an aside}
\begin{itemize}
\item What proportion of outcomes are less than or equal to 330? \\ (Answer: $10\%$)
\item What proportion of outcomes are greater than or equal to 370?\\ (Answer: $16.66\%$)
\end{itemize}
}
%----------------------------------------------------------------%
{
\textbf{What is a Histogram}

\begin{itemize}
\item A histogram is a plot that lets you discover, and show, the underlying frequency distribution (shape) of a set of continuous data.
\item This allows the inspection of the data for its underlying distribution (e.g., normal distribution), outliers, skewness, etc. 
\item Histograms should depict with the frequency or relative frequency in each ``\textbf{bin}".

\end{itemize}
}
%----------------------------------------------------------------%
{
\textbf{What is a Histogram}

\begin{itemize}
\item To construct a histogram from a continuous variable you first need to split the data into intervals, called \textbf{bins}.
\item Conventionally, the number of bins should be \textbf{\textit{near}} the square root of the sample size.
\[ N(Bins) \approx \sqrt{n} \]

\item The Bins should have equal size (i.e. same width)
\item The Bins should include the maximum and the minimum values in the data.
\item Every data point should have precisely one bins that it is supposed to go in.

\end{itemize}
}
%--------------------------------------------------------%

{
\textbf{Histograms}

For the die-throw experiment;
\begin{center}
\includegraphics[scale=0.30]{images/3aDieHist}
\end{center}

}
%----------------------------------------------------------------%
{
\textbf{Constructing Histograms}

\begin{itemize}
\item Compute an appropriate number of class intervals.
\item As a rule of thumb, the number of class intervals is usually approximately the square root of the number of observations.
\item As there are 60 observations, we would normally use 7 or 8 class intervals.
\item To save time, we will just use 5 class intervals.
\end{itemize}

}

%--------------------------------------------------------%

{
\textbf{Histograms}

\begin{center}
\includegraphics[scale=0.30]{images/3aDieHist2}
\end{center}

}
%--------------------------------------------------------%

{
\textbf{Histograms}

\begin{itemize}
\item Suppose that the experiment of throwing a die 100 times and recording the total was repeated 100,000 times.
\item \textit{(If implemented on a computer, we would call this a simulation study)}
\item The histogram of data (with a class interval width of 2) is shown on the next slide.
\item How should the shape of the histogram be described?
\item ``Bell-shaped" would be a suitable description.
\end{itemize}
}
%--------------------------------------------------------%

{
\textbf{Histograms}

\begin{center}
\includegraphics[scale=0.30]{images/3aDieHist3}
\end{center}

}
%--------------------------------------------------------%

{

\textbf{Simulation Study}
A couple of remarks about the simulation study, some of which will be relevant later on.
\begin{itemize}
% \item Approximately 76\% of the values are between 330 and 370.
\item Approximately 68.7\% of the values in the simulation study are between 332 and 367.
\item Approximately 95\% of the values are between 316 and 383.
\item $2.5\%$ of the values output are less than 316.
\item $2.5\%$ of the values study output are greater than 383.
\item 175 values are greater than or equal to 400, whereas 198 values are less than or equal to 300.
\item Results such as these are unusual, but they are not impossible.
\end{itemize}
}
\newpage
%--------------------------------------------------------%

{

\textbf{Random Variables}
A pair of dice is thrown. Let X denote the minimum of the two numbers which occur.
Find the distributions and expected value of X.
}
%-------------------------------------------------------------%
{
\textbf{Random Variables}
A fair coin is tossed four times.
Let X denote the longest string of heads.
Find the distribution and expectation of X.}
%-------------------------------------------------------------%
{\textbf{Random Variables}
A fair coin is tossed until a head or five tails occurs.
Find the expected number E of tosses of the coin.}
%-------------------------------------------------------------%
{\textbf{Random Variables}A coin is weighted so that P(H) = 0.75 and P(T ) = 0.25

The coin is tossed three times. Let X denote the number of
heads that appear.
\begin{itemize}
\item (a) Find the distribution f of X.
\item (b) Find the expectation E(X).
\end{itemize}
}

%-------------------------------------------------------------%
{
\begin{itemize}
\item Now consider an experiment with only two outcomes. Independent repeated trials of such an experiment are
called Bernoulli trials, named after the Swiss mathematician Jacob Bernoulli (1654–1705). \item The term \textbf{\emph{independent
trials}} means that the outcome of any trial does not depend on the previous outcomes (such as tossing a coin).
\item We will call one of the outcomes the ``success" and the other outcome the ``failure".
\end{itemize}
}

%-------------------------------------------------------------%
{
\begin{itemize} \item
Let $p$ denote the probability of success in a Bernoulli trial, and so $q = 1 - p$ is the probability of failure.
A binomial experiment consists of a fixed number of Bernoulli trials. \item A binomial experiment with $n$ trials and
probability $p$ of success will be denoted by
\[B(n, p)\]
\end{itemize}
}
%-------------------------------------------------------------%


\begin{verbatim}
n=60000
Y=numeric(n)
for ( i in 1:n){

X=floor(runif(100,1,7))
Y[i]=sum(X)
}

Y
hist(Y,breaks=seq(300,400,by=10),main=c("Totals of 100 Die Throws"),cex.lab=1.4,font.lab=2,xlab=c("Total Score"))

hist(Y,breaks=seq(300,400,by=20),main=c("Totals of 100 Die Throws"),cex.lab=1.4,font.lab=2,xlab=c("Total Score"))



Z=seq(1:n)
Y/Z

plot(Y/Z,type="l",col="red",main=c("Die Rolls: Running Average"),font.lab=2,ylab="Average Value", xlab=
" Number of Throws")
abline(h=3.5,col="green")


#####################################################

plot(Z,Z.y,pch=16,col="red",ylim=c(2.5,5.5),main=c("Variance"),font.lab=2,ylab=" ", xlab="X: Green  Y: Blue  Z: Red" )

points(Y,Y.y,pch=16,col="blue" )
points(X,X.y,pch=16,col="green" )
points(c(1000,1000,1000),c(3,4,5),pch=18,cex=1.2)
lines(c(1000,1000),c(2.75,5.25),lty=3)



n=100000
Y=numeric(n)
for ( i in 1:n){

X=floor(runif(100,1,7))
Y[i]=sum(X)
}

Y
hist(Y,breaks=seq(270,430,by=2),main=c("Totals of 100 Die Throws (n= 100,000)"),cex.lab=1.4,font.lab=2,xlab=c("Total Score")) 
\end{verbatim}



{
\textbf{Boxplots}
\begin{itemize}
\item The second graphical method we will be looking at today is the `box-and-whisker' plot (commonly just referred to as `boxplots')
\item The boxplots is a useful tool for assessing the distribution of a dataset, by means of a visual summary.
\item Recall the data set of the exam scores of 100 students from yesterday's class (see next slide).
\item The quartiles of the data set were $Q_1 = 42.5$, $Q_2 = 54.5$ (with $Q_2$ being the median), and $Q_3 =  65.5$ respectively.
\item The interquartile range is $Q_3 - Q_1 = 23$
\item The boxplot of the distribution is featured on the next slide.
\end{itemize}
}
%---------------------------------------------------------------------------%
{
\begin{table}[ht]
\caption{Exam results of 100 students} % title of Table
\centering % used for centering table
\begin{tabular}{|c ccc ccc ccc|} % centered columns (4 columns)\hline
\hline

13&21&22&23&24&25&26&28&29&30\\31&32&33&34&35& 36&36&36&37&38\\
39&41&41&41&42&43&44&44&44&45\\45&46&47&49&50& 51&51&52&53&53\\
53&53&53&54&54&54&54&54&54&54\\55&55&55&56&56& 56&57&57&58&59\\
62&63&63&63&63&64&64&64&64&64\\65&65&65&65&65& 66&66&66&67&69\\
71&71&72&72&73&74&75&76&76&76\\77&82&84&85&87& 88&91&91&92&99\\ \hline
\end{tabular}
\end{table}
}


\newpage
%----------------------------------------------------------------%
{
\section{Histograms}
\begin{itemize}
\item  Consider an experiment in which each student in a class of 60 rolls a die 100 times.
\item  Each score is recorded, and a total score is calculated.
\item  As the expected value of rolled die is 3.5, the expected total is 350 for each student.
\item  At the end of the experiment the students reported their totals.
\item  The totals were put into ascending order, and tabulated as follows (next slide).
\end{itemize}

}

%--------------------------------------------------------%
{
\textbf{Outcomes of die-throw experiment}
\small
\begin{center}
\begin{tabular}{|c c c c c c c c c c|}
\hline
% after \\: \hline or \cline{col1-col2} \cline{col3-col4} ...
307 & 321 & 324 & 328 & 329 & 330 & 334 & 335 & 336 &337 \\
337 & 337 & 338 & 339 & 339 & 342 & 343 & 343 & 344 &344 \\
346 & 346 & 347 & 348 & 348 & 348 & 350 & 351 & 352 &352 \\
353 & 353 & 353 & 354 & 354 & 356 & 356 & 357 & 357 &358 \\
358 & 360 & 360 & 361 & 362 & 363 & 365 & 365 & 369 &369 \\
370 & 370 & 374 & 378 & 381 & 384 & 385 & 386 & 392 &398 \\
\hline
\end{tabular}
\end{center}
\normalsize
\begin{itemize}
\item  What proportion of outcomes are less than or equal to 330? \\ (Answer: $10\%$)
\item  What proportion of outcomes are greater than or equal to 370?\\ (Answer: $16.66\%$)
\end{itemize}
}
%--------------------------------------------------------%

{
\textbf{What is a Histograms}



}
%--------------------------------------------------------%

{
\textbf{Histograms}
For the die-throw experiment;
\begin{center}
\includegraphics[scale=0.30]{images/3aDieHist}
\end{center}

}
%----------------------------------------------------------------%
{
\textbf{Constructing Histograms}
\begin{itemize}
\item  Compute an appropriate number of class intervals.
\item  As a rule of thumb, the number of class intervals is usually approximately the square root of the number of observations.
\item  As there are 60 observations, we would normally use 7 or 8 class intervals.
\item  To save time, we will just use 5 class intervals.
\end{itemize}

}

%--------------------------------------------------------%

{
\textbf{Histograms}

\begin{center}
\includegraphics[scale=0.30]{images/3aDieHist2}
\end{center}

}
%--------------------------------------------------------%

{
\textbf{Histograms}
\begin{itemize}
\item  Suppose that the experiment of throwing a die 100 times and recording the total was repeated 100,000 times.
\item  (If implemented on a computer, we would call this a simulation study)
\item  The histogram of data (with a class interval width of 2) is shown on the next slide.
\item  How should the shape of the histogram be described?
\item  ``Bell-shaped" would be a suitable description.
\end{itemize}
}
%--------------------------------------------------------%

{
\textbf{Histograms}

\begin{center}
\includegraphics[scale=0.30]{images/3aDieHist3}
\end{center}

}
%--------------------------------------------------------%



%--------------------------------------------------------%


{
\textbf{Boxplots}

\begin{center}
\includegraphics[scale=0.40]{images/3Bboxplot1}
\end{center}
}



%----------------------------------------------------------------%
{
\subsection{Histograms}
\begin{itemize}
\item Consider an experiment in which each student in a class of 60 rolls a die 100 times.
\item Each score is recorded, and a total score is calculated.
\item As the expected value of rolled die is 3.5, the expected total is 350 for each student.
\item At the end of the experiment the students reported their totals.
\item The totals were put into ascending order, and tabulated as follows (next slide).
\end{itemize}

}

%--------------------------------------------------------%
{
\subsection{Outcomes of die-throw experiment}
\small
\begin{center}
\begin{tabular}{|c c c c c c c c c c|}
\hline
% after \\: \hline or \cline{col1-col2} \cline{col3-col4} ...
307 & 321 & 324 & 328 & 329 & 330 & 334 & 335 & 336 &337 \\
337 & 337 & 338 & 339 & 339 & 342 & 343 & 343 & 344 &344 \\
346 & 346 & 347 & 348 & 348 & 348 & 350 & 351 & 352 &352 \\
353 & 353 & 353 & 354 & 354 & 356 & 356 & 357 & 357 &358 \\
358 & 360 & 360 & 361 & 362 & 363 & 365 & 365 & 369 &369 \\
370 & 370 & 374 & 378 & 381 & 384 & 385 & 386 & 392 &398 \\
\hline
\end{tabular}
\end{center}
\normalsize
\begin{itemize}
\item What proportion of outcomes are less than or equal to 330? \\ (Answer: $10\%$)
\item What proportion of outcomes are greater than or equal to 370?\\ (Answer: $16.66\%$)
\end{itemize}
}
%--------------------------------------------------------%

{
\subsection{What is a Histograms}



}
%--------------------------------------------------------%

{
\subsection{Histograms}
For the die-throw experiment;
\begin{center}
\includegraphics[scale=0.30]{images/3aDieHist}
\end{center}

}
%----------------------------------------------------------------%
{
\subsection{Constructing Histograms}
\begin{itemize}
\item Compute an appropriate number of class intervals.
\item As a rule of thumb, the number of class intervals is usually approximately the square root of the number of observations.
\item As there are 60 observations, we would normally use 7 or 8 class intervals.
\item To save time, we will just use 5 class intervals.
\end{itemize}

}

%--------------------------------------------------------%


\subsection{Histograms}

\begin{center}
\includegraphics[scale=0.30]{images/3aDieHist2}
\end{center}


%--------------------------------------------------------%

\subsection{Histograms}
\begin{itemize}
\item Suppose that the experiment of throwing a die 100 times and recording the total was repeated 100,000 times.
\item (If implemented on a computer, we would call this a simulation study)
\item The histogram of data (with a class interval width of 2) is shown on the next slide.
\item How should the shape of the histogram be described?
\item ``Bell-shaped" would be a suitable description.
\end{itemize}

%--------------------------------------------------------%

{
\subsection{Histograms}

\begin{center}
\includegraphics[scale=0.30]{images/3aDieHist3}
\end{center}

}
%--------------------------------------------------------%

{
\subsection{Simulation Study}
A couple of remarks about the simulation study, some of which will be relevant later on.
\begin{itemize}
% \item Approximately 76\% of the values are between 330 and 370.
\item Approximately 68.7\% of the values in the simulation study are between 332 and 367.
\item Approximately 95\% of the values are between 316 and 383.
\item $2.5\%$ of the values output are less than 316.
\item $2.5\%$ of the values study output are greater than 383.
\item 175 values are greater than or equal to 400, whereas 198 values are less than or equal to 300.
\item Results such as these are unusual, but they are not impossible.
\end{itemize}
}

\newpage


\section{Categorical Data}
\subsection{Visualising Categorical Data}
{ \textbf{Visualising Categorical Data}}
We first count the number of entries in each category - the frequencies - and construct a { frequency table}.\\[0.4cm]
A { bar chart} is simply a graph with the frequencies (or relative frequencies) on the y-axis and the category labels on the x-axis.\\[0.6cm]
Consider the following example:\\[0.2cm]
In 2011 a market researcher carried out an online survey with the intention of discovering the market share of various mobile devices. Participants were asked tick a box indicating the mobile device that they use: ``Android'', ``Apple'', ``BlackBerry'', ``Windows'' or ``Other''. In total 500 individuals were surveyed and a frequency table was constructed (see next slide).






\subsection{Categorical Data: Frequency Table}
{ \textbf{Categorical Data: Frequency Table}\\[-1cm]}
\begin{center}
Market Share 2011: Ordered highest to lowest frequency\\[0.1cm]
\begin{tabular}{|c|r|r|}
\hline
&&\\[-0.4cm]
Category   & Frequency & Relative Frequency \\
&&\\[-0.5cm]
\hline
&&\\[-0.4cm]
Android    &   174     & $\tfrac{174}{500} = 0.348$ \\[0.2cm]
Other      &   138     & $\tfrac{138}{500} = 0.276$ \\[0.2cm]
Apple      &   107     & $\tfrac{107}{500} = 0.214$ \\[0.2cm]
BlackBerry &    74     & $\tfrac{74}{500} = 0.148$ \\[0.2cm]
Windows    &     7     & $\tfrac{7}{500} = 0.014$ \\[0.2cm]
\hline
&&\\[-0.4cm]
\multicolumn{1}{|r|}{Total:} & $n = 500$ & $\tfrac{500}{500} = 1.000$ \\[0.1cm]
\hline
\end{tabular}
\end{center}
\begin{itemize}\itemsep0.2cm
\item The symbol for the total sample size is {\boldmath$n$} - we will use this throughout the course.
\item The relative frequencies (or proportions) add to 1.00. Also, these serve as estimates of the true \emph{population proportions}.
\end{itemize}





\subsection{Categorical Data: Bar Chart (Frequency)}
%{ \textbf{Categorical Data: Bar Chart (Frequency)}\\[-1.1cm]}
%\begin{center}
%\includegraphics[width=0.8\textwidth, trim = 0.0cm 0.5cm 0.3cm 1cm, clip]{MobileDevice-Rel}
%\end{center}
\[IMAGE\]
\begin{itemize}\itemsep0.2cm
\item Note that there are { gaps} between the various categories.
\end{itemize}


\subsection{Categorical Data: Bar Chart (Relative Frequency)}
%{ \textbf{Categorical Data: Bar Chart (Relative Frequency)}\\[-1.1cm]}
%\begin{center}
%\includegraphics[width=0.8\textwidth, trim = 0.0cm 0.5cm 0.3cm 1cm, clip]{MobileDevice-Rel}
%\end{center}
\[IMAGE\]
\begin{itemize}\itemsep0.2cm
\item Same picture but using relative frequency (see y-axis).
\end{itemize}


%%\subsection{R Code: Bar Chart}
%%{ \textbf{R Code: Bar Chart}\\[-1.1cm]}
%%The R code used to draw a bar chart is:\\[0.1cm]
%%\begin{tabular}{|l|}
%%\hline
%%\texttt{freq = c(174,138,107,74,7)}\\
%%\texttt{mobile = c("Android","Other","Apple","BlackBerry",}\\
%%\hspace{2.6cm}\texttt{"Windows")}\\
%%\texttt{barplot(freq, names=mobile)}\\
%%\hline
%%\multicolumn{1}{c}{}\\[-0.1cm]
%%\end{tabular}
%%You should \emph{always} label the axes:\\[0.1cm]
%%\begin{tabular}{|l|}
%%\hline
%%\texttt{barplot(freq, names=mobile, xlab="Mobile Device",}\\
%%\hspace{1.9cm}\texttt{ylab="Frequency")}\\
%%\hline
%%\multicolumn{1}{c}{}\\[-0.1cm]
%%\end{tabular}
%%Some aesthetic improvements:\\[0.1cm]
%%\begin{tabular}{|l|}
%%\hline
%%\texttt{barplot(freq, names=mobile, xlab="Mobile Device",}\\
%%\hspace{1.9cm}\texttt{ylab="Frequency",density=20)}\\
%%\texttt{abline(h=0)}\\
%%\hline
%%\multicolumn{1}{c}{}\\[-0.1cm]
%%\end{tabular}
%%Run \boxed{\text{\texttt{?barplot}}} for more details.
%%








%----------------------------------------------------------------%
{
\subsection{Histograms}
\begin{itemize}
\item Consider an experiment in which each student in a class of 60 rolls a die 100 times.
\item Each score is recorded, and a total score is calculated.
\item As the expected value of rolled die is 3.5, the expected total is 350 for each student.
\item At the end of the experiment the students reported their totals.
\item The totals were put into ascending order, and tabulated as follows (next slide).
\end{itemize}

}

%--------------------------------------------------------%
{
\subsection{Outcomes of die-throw experiment}
\small
\begin{center}
\begin{tabular}{|c c c c c c c c c c|}
\hline
% after \\: \hline or \cline{col1-col2} \cline{col3-col4} ...
307 & 321 & 324 & 328 & 329 & 330 & 334 & 335 & 336 &337 \\
337 & 337 & 338 & 339 & 339 & 342 & 343 & 343 & 344 &344 \\
346 & 346 & 347 & 348 & 348 & 348 & 350 & 351 & 352 &352 \\
353 & 353 & 353 & 354 & 354 & 356 & 356 & 357 & 357 &358 \\
358 & 360 & 360 & 361 & 362 & 363 & 365 & 365 & 369 &369 \\
370 & 370 & 374 & 378 & 381 & 384 & 385 & 386 & 392 &398 \\
\hline
\end{tabular}
\end{center}
\normalsize
\begin{itemize}
\item What proportion of outcomes are less than or equal to 330? \\ (Answer: $10\%$)
\item What proportion of outcomes are greater than or equal to 370?\\ (Answer: $16.66\%$)
\end{itemize}
}
%--------------------------------------------------------%

{
\subsection{What is a Histograms}



}
%--------------------------------------------------------%

{
\subsection{Histograms}
For the die-throw experiment;
\begin{center}
\includegraphics[scale=0.30]{images/3aDieHist}
\end{center}

}
%----------------------------------------------------------------%
{
\subsection{Constructing Histograms}
\begin{itemize}
\item Compute an appropriate number of class intervals.
\item As a rule of thumb, the number of class intervals is usually approximately the square root of the number of observations.
\item As there are 60 observations, we would normally use 7 or 8 class intervals.
\item To save time, we will just use 5 class intervals.
\end{itemize}

}

%--------------------------------------------------------%


\subsection{Histograms}

\begin{center}
\includegraphics[scale=0.30]{images/3aDieHist2}
\end{center}


%--------------------------------------------------------%

\subsection{Histograms}
\begin{itemize}
\item Suppose that the experiment of throwing a die 100 times and recording the total was repeated 100,000 times.
\item (If implemented on a computer, we would call this a simulation study)
\item The histogram of data (with a class interval width of 2) is shown on the next slide.
\item How should the shape of the histogram be described?
\item ``Bell-shaped" would be a suitable description.
\end{itemize}

%--------------------------------------------------------%

{
\subsection{Histograms}

\begin{center}
\includegraphics[scale=0.30]{images/3aDieHist3}
\end{center}

}
%--------------------------------------------------------%

{
\subsection{Simulation Study}
A couple of remarks about the simulation study, some of which will be relevant later on.
\begin{itemize}
% \item Approximately 76\% of the values are between 330 and 370.
\item Approximately 68.7\% of the values in the simulation study are between 332 and 367.
\item Approximately 95\% of the values are between 316 and 383.
\item $2.5\%$ of the values output are less than 316.
\item $2.5\%$ of the values study output are greater than 383.
\item 175 values are greater than or equal to 400, whereas 198 values are less than or equal to 300.
\item Results such as these are unusual, but they are not impossible.
\end{itemize}
}

%--------------------------------------------------------%

\section{Graphical Methods}
\begin{itemize}
\item Expected value and variance of discrete random variables
\item Graphical methods
\begin{itemize}
\item Frequency tables
\item Histograms
\end{itemize}
\item Some remarks that relate to future material
\end{itemize}


	%------------------------------------------------------------------%
	{
		\textbf{Bar plots}
		
		\begin{itemize} \item  A bar plot displays the frequency (or relative frequency) for all observations of a discrete random variable. \item  A bar plot is much like a histogram, in that the heights of columns represent the frequency (or relative frequency) of each outcome.
			\item  Each outcome of a random experiment corresponds to one and only one column of the bar plot.
			\item  A bar plot differs from a histogram in that the columns are distinct and separated from each other by a small distance.
		\end{itemize}
	}
	%------------------------------------------------------------------%
	{
		\textbf{Bar plots}
		
		Suppose we roll a die 300 times, and obtain the following results
		
		\begin{center}
			\begin{tabular}{|c|c|c|c|c|c|c|}
				\hline
				
				Outcome & 1 & 2 & 3 & 4 & 5 & 6 \\
				Frequency & 59 &41 &39 &52 &57 &52  \\
				Rel. Freq & 0.196 & 0.136 & 0.130 & 0.173 & 0.190 & 0.173\\
				\hline
			\end{tabular}
		\end{center}
		
		On the next slide is the bar plot of the relative frequencies of the outcomes of die throw experiment.
		Included on the bar plot is the theoretical probability of each outcome. As each outcome is equally probable, this is just a straight line.\\ \bigskip
		Minor deviations from the theoretical probability can often be assumed to be as a result of random error. In the case of large deviations, there may be a flawed assumption about the theoretical probabilities.
	}
	
	%--------------------------------------------------------%
	
	{
		\textbf{Relative Frequency Bar Plots}
		
		\begin{center}
			\includegraphics[scale=0.40]{3Bbarplot}
		\end{center}
	}
	
	%--------------------------------------------------------%
	
	{
		\textbf{Relative Frequency Bar Plots}
		
		\begin{itemize}
			\item  Just as bar plots can be used to graphically depict observed relative frequencies, they can be used to
			depict the theoretical probabilities of each outcome.
			\item  We will be using bar plots to visualize the theoretical probabilities of outcomes of discrete random variables.
			\item  For this module, bar plots are assumed to be used for this purpose, unless it is clearly expressed otherwise.
			\item  On the next slide is a bar plot of the probabilities of each outcome of a dice throw.
		\end{itemize}
		
	}
	
	
	%--------------------------------------------------------%
	
	{
		\textbf{Bar Plots}
		\begin{center}
			\includegraphics[scale=0.40]{3Bbarplot3}
		\end{center}
	}
	
	{
		\textbf{Bar Plots}
		
		\begin{itemize}
			\item  Bar plots are useful in that they visualize `events'.
			\item  Consider the event where either a `4' or a `5' is thrown.
			\item  The relevant columns for this event are shaded (next slide).
			\item  We will be using bar plots for depicting specific events in upcoming material
		\end{itemize}
		
	}
	%--------------------------------------------------------%
	
	{
		\textbf{Bar Plots}
		
		\begin{center}
			\includegraphics[scale=0.40]{3Bbarplot2}
		\end{center}
	}


%--------------------------------------------------------%
\newpage
\begin{verbatim}
%----R Code ----

n=60000
Y=numeric(n)
for ( i in 1:n){

X=floor(runif(100,1,7))
Y[i]=sum(X)
}

Y
hist(Y,breaks=seq(300,400,by=10),main=c("Totals of 100 Die Throws"),cex.lab=1.4,font.lab=2,xlab=c("Total Score"))

hist(Y,breaks=seq(300,400,by=20),main=c("Totals of 100 Die Throws"),cex.lab=1.4,font.lab=2,xlab=c("Total Score"))



Z=seq(1:n)
Y/Z

plot(Y/Z,type="l",col="red",main=c("Die Rolls: Running Average"),font.lab=2,ylab="Average Value", xlab=
" Number of Throws")
abline(h=3.5,col="green")


\end{verbatim}
\begin{verbatim}

plot(Z,Z.y,pch=16,col="red",ylim=c(2.5,5.5),main=c("Variance"),font.lab=2,ylab=" ", xlab="X: Green  Y: Blue  Z: Red" )

points(Y,Y.y,pch=16,col="blue" )
points(X,X.y,pch=16,col="green" )
points(c(1000,1000,1000),c(3,4,5),pch=18,cex=1.2)
lines(c(1000,1000),c(2.75,5.25),lty=3)



n=100000
Y=numeric(n)
for ( i in 1:n){

X=floor(runif(100,1,7))
Y[i]=mean(X)
}

Y
hist(Y,breaks=seq(270,430,by=2),main=c("Mean of 100 Die Throws (n= 100,000)"),cex.lab=1.4,font.lab=2,xlab=c("Mean of 100 throws")) 
\end{verbatim}

\subsection{Question 4}
{ \textbf{Question 4}\\[-0.8cm]}
The following year (2012) a survey found that 359 individuals used Android, 81 used Apple, 18 used BlackBerry, 18 used Windows and 24 used other devices.\\[0.2cm]
\begin{enumerate}
\item What is the value of $n$\,?
\item Construct a frequency table (ordered highest to lowest frequency) and include a column with relative frequencies.
\item Estimate the proportion of individuals who use either Android or Apple devices.

\item Estimate the proportion of individuals who use other devices. What symbol would we use for this proportion?
\item What is the \textbf{true} proportion of individuals who use other devices? What symbol would we use for this proportion?
\item Draw the bar chart.
\item Comment on how the market has changed since 2011.
\end{enumerate}


\newpage

\end{document}
