



%--------------------------------------------------------%
{
	{Range}

\begin{itemize}
\item The range of a sample (or a data set) is a measure of the spread or the dispersion of the observations. \item It is the difference between the largest and the smallest observed value of some quantitative characteristic and is very easy to calculate.
\smallskip
\item A great deal of information is ignored when computing the range since only the largest and the smallest data values are considered; the remaining data are ignored.
\smallskip
\item The range value of a data set is greatly influenced by the presence of just one unusually large or small value in the sample (outlier).
\end{itemize}

\t{Example}


The range of $\{65,73,89,56,73,52,47\}$ is $ 89-47 = 42$.

% If the highest score in a 1st year statistics exam was 98 and the lowest 48, then the range would be 98-48 = 50.

}



%--------------------------------------------------------%

\begin{itemize}
\item The range is a very simple measure of dispersion.
\item It is simply the difference between the maximum and minimum values.
\end{itemize}

Consider the following data set
\[ X= \{3,5,6,7,8,9\}\]

\[\mbox{Range} =  Max - Min = 9-3 = 6 \]









\begin{itemize}
\item The range of a set of data is the difference between the highest and lowest values in the data set.
\item Consider the following data set
\[  \{  39,  23,  34,  41,  37,  27,  44 \}\]
\item The highest value (i.e. the maximum) is 44.
\item The lowest value (i.e. the minimum) is 23.
\item The range is the difference is between these two numbers: 
\[ \mbox{Range } = 44 - 23 = 21. \] 
\end{itemize}


\subsection{Range}
Range = maximum point  - minimum point

Since it is calculated using only 2 values, it tells us nothing about the data in between these two values and therefore, conveys the least information.



\end{document}
