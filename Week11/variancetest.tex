	\documentclass[a4paper,12pt]{article}
%%%%%%%%%%%%%%%%%%%%%%%%%%%%%%%%%%%%%%%%%%%%%%%%%%%%%%%%%%%%%%%%%%%%%%%%%%%%%%%%%%%%%%%%%%%%%%%%%%%%%%%%%%%%%%%%%%%%%%%%%%%%%%%%%%%%%%%%%%%%%%%%%%%%%%%%%%%%%%%%%%%%%%%%%%%%%%%%%%%%%%%%%%%%%%%%%%%%%%%%%%%%%%%%%%%%%%%%%%%%%%%%%%%%%%%%%%%%%%%%%%%%%%%%%%%%
\usepackage{eurosym}
\usepackage{framed}
\usepackage{vmargin}
\usepackage{amsmath}
\usepackage{graphics}
\usepackage{epsfig}
\usepackage{subfigure}
\usepackage{enumerate}
\usepackage{fancyhdr}

\setcounter{MaxMatrixCols}{10}
%TCIDATA{OutputFilter=LATEX.DLL}
%TCIDATA{Version=5.00.0.2570}
%TCIDATA{<META NAME="SaveForMode"CONTENT="1">}
%TCIDATA{LastRevised=Wednesday, February 23, 201113:24:34}
%TCIDATA{<META NAME="GraphicsSave" CONTENT="32">}
%TCIDATA{Language=American English}

\pagestyle{fancy}
\setmarginsrb{20mm}{0mm}{20mm}{25mm}{12mm}{11mm}{0mm}{11mm}
\lhead{MS4222} \rhead{Kevin O'Brien} \chead{Statistical Computing} %\input{tcilatex}

\begin{document}	
\section*{Test for Equality of Variance with \texttt{R}}
	\begin{itemize}
		\item In this procedure, we determine whether or not two populations have the same variance.
		\item The assumption of equal variance of two populations underpins several inference procedures. This assumption is tested by comparing the variance of samples taken from both populations.
\textit{ We will not get into too much detail about that, but concentrate on how to assess equality of variance.}
		\item The null and alternative hypotheses are as follows:
		\[ H_0: \sigma^2_1 = \sigma^2_2 \]
		\[ H_1: \sigma^2_1 \neq \sigma^2_2 \]

		\item When using \texttt{R} it would be convenient to consider the null and alternative in terms of variance ratios.
		\item Two data sets have equal variance if the variance ratio is 1.

	\[ H_0: \sigma^2_1 / \sigma^2_2 = 1 \]
	\[ H_1: \sigma^2_1 / \sigma^2_2 \neq 1 \]
	
	%----------------------------------------%
	% - x=c(14,13,16,20,12,18,11,09,13,11)
	% - y=c(15,13,18,20,10,17,23,11,10)
	%----------------------------------------%

\item You would be required to compute the test statistic for this procedure.
\item The test statistic is the ratio of the variances for both data sets.
	\[ TS = \frac{s^2_x}{s^2_y} \]
\item The standard deviations would be provided in the question. Suppose we have the following values: \begin{itemize}
		\item[$\ast$] Sample standard deviation for data set $x$ = 3.40.
		\item[$\ast$] Sample standard deviation for data set $y$ = 4.63.
	\end{itemize}
\item To compute the test statistic.
	\[ TS = \frac{3.40^2}{4.63^2} = \frac{11.56}{21.43} = 0.5394 \]
		\end{itemize}
\newpage
	\begin{framed}
		\begin{verbatim}
		> var.test(x,y)
		
		F test to compare two variances
		
		data:  x and y
		
		F = 0.5394, num df = 9, denom df = 8, p-value = 0.3764
		
		alternative hypothesis: 
		  true ratio of variances is not equal to 1
		
		95 percent confidence interval:
		  0.1237892 2.2125056
		
		sample estimates:
		  ratio of variances
		  0.5393782
		\end{verbatim}
	\end{framed}		
	

	\begin{itemize}
		\item The $p-$value is 0.3764 (top right). The threshold is $\alpha/2 = (0.05/2) = 0.0250$.
		\item We fail to reject the null hypothesis. \medskip
		\item We don't have evidence of a significant difference in sample variances. Therefore we can use the assumption that both populations have equal variance. \medskip
		\item Additionally the $95\%$ confidence interval (0.1237, 2.2125) contains the expected value of the variance ration under the null hypothessi i.e. 1.
	\end{itemize}
	
\end{document}
