\documentclass[a4paper,12pt]{article}
%%%%%%%%%%%%%%%%%%%%%%%%%%%%%%%%%%%%%%%%%%%%%%%%%%%%%%%%%%%%%%%%%%%%%%%%%%%%%%%%%%%%%%%%%%%%%%%%%%%%%%%%%%%%%%%%%%%%%%%%%%%%%%%%%%%%%%%%%%%%%%%%%%%%%%%%%%%%%%%%%%%%%%%%%%%%%%%%%%%%%%%%%%%%%%%%%%%%%%%%%%%%%%%%%%%%%%%%%%%%%%%%%%%%%%%%%%%%%%%%%%%%%%%%%%%%
\usepackage{eurosym}
\usepackage{vmargin}
\usepackage{amsmath}
\usepackage{graphics}
\usepackage{epsfig}
\usepackage{subfigure}
\usepackage{enumerate}
\usepackage{fancyhdr}

\setcounter{MaxMatrixCols}{10}
%TCIDATA{OutputFilter=LATEX.DLL}
%TCIDATA{Version=5.00.0.2570}
%TCIDATA{<META NAME="SaveForMode"CONTENT="1">}
%TCIDATA{LastRevised=Wednesday, February 23, 201113:24:34}
%TCIDATA{<META NAME="GraphicsSave" CONTENT="32">}
%TCIDATA{Language=American English}

\pagestyle{fancy}
\setmarginsrb{20mm}{0mm}{20mm}{25mm}{12mm}{11mm}{0mm}{11mm}
\lhead{MS4222} \rhead{Kevin O'Brien} \chead{Akaike's Information Criterion} %\input{tcilatex}

\begin{document}
\section*{Akaike's Information Criterion}

\begin{itemize}
\item The AIC is a model selection metric often used in statistics. 
\item Akaike's information criterion is a measure of the goodness of fit of an estimated statistical model.

\item The AIC was developed by Hirotsugu Akaike under the name of ``An Information Criterion" in 1971.

\item The AIC is calculated using the "likelihood function" and the number of parameters. The likelihood value is generally given in code output, as a complement to the AIC.

\item It is computed using the R command
\texttt{\textbf{AIC()}}. The candidate model with the smallest AIC value is considered preferable.

\item The AIC is a "model selection" tool i.e. a method of comparing two 
or more candidate models.
\item The AIC methodology attempts to find the model that best explains the data with a minimum of parameters.
(i.e. in keeping with the Law of parsimony)

\item Given a data set, several competing models may be ranked according to their AIC, with the one having the lowest AIC being the best. \textit{(A difference in AIC values of less than two is considered negligible)}
\end{itemize}


\end{document}
