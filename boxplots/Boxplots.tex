\documentclass[a4paper,12pt]{article}
%%%%%%%%%%%%%%%%%%%%%%%%%%%%%%%%%%%%%%%%%%%%%%%%%%%%%%%%%%%%%%%%%%%%%%%%%%%%%%%%%%%%%%%%%%%%%%%%%%%%%%%%%%%%%%%%%%%%%%%%%%%%%%%%%%%%%%%%%%%%%%%%%%%%%%%%%%%%%%%%%%%%%%%%%%%%%%%%%%%%%%%%%%%%%%%%%%%%%%%%%%%%%%%%%%%%%%%%%%%%%%%%%%%%%%%%%%%%%%%%%%%%%%%%%%%%
\usepackage{eurosym}
\usepackage{vmargin}
\usepackage{amsmath}
\usepackage{graphics}
\usepackage{epsfig}
\usepackage{subfigure}
\usepackage{framed}
\usepackage{enumerate}
\usepackage{fancyhdr}

\setcounter{MaxMatrixCols}{10}
%TCIDATA{OutputFilter=LATEX.DLL}
%TCIDATA{Version=5.00.0.2570}
%TCIDATA{<META NAME="SaveForMode"CONTENT="1">}
%TCIDATA{LastRevised=Wednesday, February 23, 201113:24:34}
%TCIDATA{<META NAME="GraphicsSave" CONTENT="32">}
%TCIDATA{Language=American English}

\pagestyle{fancy}
\setmarginsrb{20mm}{0mm}{20mm}{25mm}{12mm}{11mm}{0mm}{11mm}
\lhead{MS4222} \rhead{Kevin O'Brien} \chead{Box and Whisker Plots} %\input{tcilatex}

\begin{document}
%---------------------------------------------------------------------------%

{
\section*{Boxplots}
\begin{itemize}
\item An important graphical method is the `box-and-whisker' plot (commonly just referred to as `boxplots')
\item The boxplot is a useful tool for assessing the distribution of a dataset, by means of a visual summary.
\item Suppose we have the exam scores of 100 students (see the table below).
\item The quartiles of the data set were $Q_1 = 42.5$, $Q_2 = 54.5$ (with $Q_2$ being the median), and $Q_3 =  65.5$ respectively.
\item The interquartile range is $Q_3 - Q_1 = 23$
\item The boxplot of the distribution is featured below the table.
\end{itemize}

\begin{table}[ht]
\caption{Exam results of 100 students} % title of Table
\centering % used for centering table
\begin{tabular}{|c ccc ccc ccc|} % centered columns (4 columns)\hline
\hline

13&21&22&23&24&25&26&28&29&30\\31&32&33&34&35& 36&36&36&37&38\\
39&41&41&41&42&43&44&44&44&45\\45&46&47&49&50& 51&51&52&53&53\\
53&53&53&54&54&54&54&54&54&54\\55&55&55&56&56& 56&57&57&58&59\\
62&63&63&63&63&64&64&64&64&64\\65&65&65&65&65& 66&66&66&67&69\\
71&71&72&72&73&74&75&76&76&76\\77&82&84&85&87& 88&91&91&92&99\\ \hline
\end{tabular}
\end{table}


\begin{center}
\includegraphics[scale=0.50]{images/3Bboxplot1}
\end{center}

\textbf{Boxplots}
\begin{itemize}
\item The boxplot is a visual summary containing important aspects of a distribution. \item The main component of the plot, the `\textbf{\emph{box}}', stretches from the \textbf{\emph{lower hinge}}, defined as $Q_1$, to the \textbf{\emph{upperhinge}}, defined as $Q_3$ .
\item The median is shown as a line across the box.
\item Therefore the box contains the middle half of the scores in the distribution.
\item  1/4 of the distribution is between the median line and the upper hinge. Similary 1/4 of the distribution is between the median line and the lower hinge.
\end{itemize}

\subsection*{Whiskers}
\begin{itemize}
\item On either side of the box are the \textbf{\emph{whiskers}}.
\item To find where to place the whiskers, we must first compute the location of the \textbf{\emph{fences}}, and determine whether or not there are any \textbf{\emph{outliers}} present.
\item Firstly, we must compute the location of the \textbf{\emph{lower fence}}.
\[ \mbox{ Lower Fence}  = Q_1 - 1.5 \times IQR \]
\item For our example, the lower fence is
\[ \mbox{ Lower Fence}  = 42.5 - 1.5 \times 23  = 42.5 - 34.5 = 8 \]

\end{itemize}

\subsection*{Lower Fences}
\begin{itemize}
\item The lower fence is used to determine whether there are any outliers in the lower half of the data set.
\item If there is any observed value less than the lower fence, it is considered an outlier.
\item The first whisker is drawn at the location of the lowest value that is not considered an outlier.
\item If no values are considered outliers, then the whisker is drawn at the location of the smallest value of the dataset.
\item For our dataset, the lowest value is 13, which is not less than the lower fence.
\item Therefore we draw the first whisker , a vertical line, at this location.
\item A horizontal line is drawn connecting the location of this whisker to $Q_1$.

\item Any value considered to be an outlier should be indicated with an asterisk or a small circle.
\item We will see an example of a boxplot with outliers in due course.
\end{itemize}

\subsection*{Upper Fences}
\begin{itemize}
\item Now we must compute the location of the \textbf{\emph{upper fence}}.
\[ \mbox{ Upper Fence}  = Q_3 + 1.5 \times IQR \]
\item For our example, the upper fence is
\[ \mbox{ Upper Fence}  = 65.5  + 1.5 \times 23  = 65.5 + 34.5 = 100 \]

\item The upper fence is used to determine whether there are any outliers in the upper half of the data set.
\item If there is any observed value greater than the upper fence, it is considered an outlier.
\item The second whisker is drawn at the location of the highest value that is not considered an outlier.
\item If no values are considered outliers, then the whisker is drawn at the location of the highest value of the dataset.
\item For our dataset, the highest value is 99, which is less than the upper fence.
\item Therefore we draw the second whisker , a vertical line, at this location.
\item A horizontal line is drawn connecting the location of this whisker to $Q_3$.
\end{itemize}

\subsection*{Sensible Values For Fences}
\begin{itemize}
\item If you do not get a sensible value for either the upper or lower fence, you can replace it with the nearest sensible value
\item For example, suppose we got a negative lower fence value. It does not make sense to get a negative score in an exam.
\item In this case, we could replace the value with a value of $0$.
\item similarly for the upper fence: any fence value greater than 100 should be replaced with the value of 100.
\end{itemize}

\subsection*{Comparing Two or More Data Sets}
\begin{itemize}
\item Boxplots are very useful in comparing the distributions of two or more data sets.
\item Recall the experiment of 60 students, each throwing a die 100 times.
\item Suppose they perform this experiment twice, firstly with a fair die, and then with a crooked die.
\item (The probability of the outcomes from the crooked die are as per yesterday's class).
\item Boxplots can use used to compare the distribution of the outcomes of both experiments.
\end{itemize}

\newpage

\begin{center}
\includegraphics[scale=0.40]{images/3Bboxplot2}
\end{center}
%=======================================================================
\section*{Summary: Constructing Boxplots}
To construct a boxplot

\begin{enumerate}
\item Calculate the first quartiel $Q_1$, the median, the third quartile $Q_3$ and the IQR.

\item Draw a horizontal line to represent the scale of measurement.

\item Draw a box just above  the line with the right and left ends at Q1 and Q3.

\item Draw a line through the box at the location of the median.


\item To detect outliers you need to determine a lower fence and an upper fence.
\begin{itemize}
\item[$\bullet$] Lower fence : $LF = Q_1 - 1.5 IQR$ 
\item[$\bullet$] Upper fence : $UF = Q_3 + 1.5 IQR$ 
\end{itemize}
% \textbf{Remark} Sometimes you would get invalid values for LF and UF with these calculations (e.g. negative heights). You may replace values for LF and UF with the nearest valid value.

\item Any values below the lower fence or above the upper fence are classes as outliers.


\item To finish the boxplot
\begin{itemize}
\item[$\bullet$]Mark any outliers with an asterisk ($\ast$) on the graph.
\item[$\bullet$]Extend horizontal lines, called whiskers, from the ends of the box to the smallest and largest values that are not outliers.
\item[$\bullet$](Remark – a variation is to extend to the lower and upper fences)
\end{itemize}
\end{enumerate}













\end{document}

