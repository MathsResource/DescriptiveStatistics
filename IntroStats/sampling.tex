
	\documentclass[a4paper,12pt]{article}
%%%%%%%%%%%%%%%%%%%%%%%%%%%%%%%%%%%%%%%%%%%%%%%%%%%%%%%%%%%%%%%%%%%%%%%%%%%%%%%%%%%%%%%%%%%%%%%%%%%%%%%%%%%%%%%%%%%%%%%%%%%%%%%%%%%%%%%%%%%%%%%%%%%%%%%%%%%%%%%%%%%%%%%%%%%%%%%%%%%%%%%%%%%%%%%%%%%%%%%%%%%%%%%%%%%%%%%%%%%%%%%%%%%%%%%%%%%%%%%%%%%%%%%%%%%%
\usepackage{eurosym}
\usepackage{vmargin}
\usepackage{framed}
\usepackage{amsmath}
\usepackage{graphics}
\usepackage{epsfig}
\usepackage{subfigure}
\usepackage{enumerate}
\usepackage{fancyhdr}

\setcounter{MaxMatrixCols}{10}
%TCIDATA{OutputFilter=LATEX.DLL}
%TCIDATA{Version=5.00.0.2570}
%TCIDATA{<META NAME="SaveForMode"CONTENT="1">}
%TCIDATA{LastRevised=Wednesday, February 23, 201113:24:34}
%TCIDATA{<META NAME="GraphicsSave" CONTENT="32">}
%TCIDATA{Language=American English}

\pagestyle{fancy}
\setmarginsrb{20mm}{0mm}{20mm}{25mm}{12mm}{11mm}{0mm}{11mm}
\lhead{MS4222} \rhead{Kevin O'Brien} \chead{Sampling} %\input{tcilatex}

\begin{document}
\section*{Statistical Inference : Definitions}
\subsection*{What are populations?}
\begin{itemize}
	\item A \textbf{\emph{population}} consists of an entire set of objects, observations, or scores that have something in common.
	For example, a population might be defined as students in a university.
	
	\item Some populations are only hypothetical. Consider an experiment where a die is thrown 100 times and the sum of the scores was recorded.
	\item The researcher might define a population as the sums that would result if this experiment was repeated an infinite number of times.
	\item The population is hypothetical in the sense that it is not reasonable to repeat this experiment indefinitely.
	\item The distribution of a population can be described by several parameters such as the mean and standard deviation.
	
\end{itemize}

\subsection*{What are sample?}
\begin{itemize}
	
	\item A sample is a subset of a population.
	\item Suppose we are interested in some characteristic of a population ( e.g. amount of time spent on the internet)
	\item Since it is usually impractical to test every member of a population, a sample from the population is typically
	the best approach available.
	\item To be properly representative of a population, a sample should be both \textbf{\emph{ random}} and sufficiently large.
	
	
\end{itemize}

\subsection*{Random Sampling}
\begin{itemize}
	
	\item In random sampling, each item or element of the population has an equal chance of being chosen at each draw.
	\item A sample is random if the method for obtaining the sample meets the criterion of randomness
	(each element having an equal chance at each draw).
	
	
\end{itemize}

\subsection*{Biased Sampling}
\begin{itemize}
	
	
	\item A biased sample is one in which the method used to create the sample results
	in samples that are systematically different from the population.
	
	
	\item For instance, consider a market research project on attitudes of attendees towards an event they attended.
	
	\item Collecting the data by publishing a questionnaire and asking people to fill it out and
	send it in would produce a biased sample.
	
	\item People interested enough to spend their time and energy filling out and sending in the questionnaire
	are likely to have different attitudes about the event than those not taking the time to fill out the questionnaire.
	
\end{itemize}





%----------------------------------------------------%

\subsection*{What is a parameter?}
\begin{itemize}
	\item A parameter is a numerical quantity measuring some aspect of a population of scores.
	\item The population mean $\mu$ and population variance $\sigma^2$ are commonly used parameters.
	\item Another commonly used parameter is the population proportion $\pi$.
	\item (Remark : greek letters are used to designate parameters.)
	\item Parameters are rarely known and are usually estimated by \textbf{\emph{statistics}} computed from samples.
\end{itemize}


%----------------------------------------------------%

\subsection*{What is a statistic?}
\begin{itemize}
	\item The most common use of the word `statistics'is for describing a wide range of techniques and procedures for analyzing, interpreting and displaying data.
	\item In a second usage, a ``statistic" is defined as a numerical quantity (such as the sample mean) calculated from a sample.
	\item Sample mean $\bar{x}$ and sample standard deviation $s$ are types of statistics.
	\item These statistics are used to estimate population parameters.
\end{itemize}


%----------------------------------------------------%

\subsection*{Statistical Inference Estimators}
\begin{itemize}
	\item Three important attributes of statistics as estimators are: \begin{itemize} \large
		\item unbiasedness, \item consistency,  \item  relative efficiency.\end{itemize}
	\item A statistic is unbiased if, in the long run, it's value is reasonably close to the parameter it is estimating.
	\item An estimator is consistent if the estimator tends to get closer to the parameter it is estimating as the sample size increases.

	
	\item The efficiency of a statistic is the degree to which the statistic is stable from sample to sample.
	\item That is, the less subject to sampling fluctuation a statistic is, the more efficient it is.
	\item \textbf{\emph{Sampling fluctuation}} refers to the extent to which a statistic takes on different values with different samples.
	\item That is, it refers to how much the statistic's value fluctuates from sample to sample.
	
\end{itemize}


%----------------------------------------------------%

\subsection*{ Inferential Statistics}
\begin{itemize}
	\item Inferential statistics are used to draw inferences about a population from a sample.
	\item Consider an experiment in which 10 subjects who performed a task after 24 hours of sleep
	deprivation scored 12 points lower than 10 subjects who performed after a normal night's sleep.
	\item Is the difference real or could it be due to chance?
	\item How much larger could the real difference be than the 12 points found in the sample?
	\item These are the types of questions answered by inferential statistics.
\end{itemize}

\subsection*{Bias in Sampling}
\begin{itemize}
\item If there is a tendency for a certain group of the population to be omitted from the sample or if people who refuse to co-operate form a group which is, in some way, different to the population, we have what is called a biased sample. 
\item 
We often use the word bias but its definition in the context of sampling is a systematic tendency to overestimate or underestimate the population parameter of interest. 
\item 
For the election example, if my sample consists solely of people from a disadvantaged area with high unemployment, I may under-estimate the level of support for a particular political party. 
\item
How can we eliminate bias? Bias can be eliminated by taking a random sample. This is a sample where everyone in the population has the same chance of getting into the sample and the fact that one individual has got into the sample does not affect the chances of another individual getting into the sample i.e. everyone has an independent and equal chance of being included in the sample. This method of sampling is called simple random sampling.
\end{itemize}
\end{document}

